% Mudar o titulo do comando abstract
\renewcommand{\abstractname}{Preâmbulo}

\begin{abstract}\label{Preâmbulo}
Hoje em dia  assistimos à tentativa dissimulada de supressão global dos nossos direitos, liberdades e garantias. De acordo com a legislação actualmente em vigor, qualquer pessoa que, de boa fé e sem quaisquer fins lucrativos, ouse partilhar qualquer obra cujo autor não esteja morto e enterrado há mais de 70 anos, está a cometer um crime.

Esta situação tem levado a um atraso no desenvolvimento que impede a evolução científica e cultural. Tudo por causa de um sistema de Copyright antiquado, não adequado às novas tecnologias, e devido a um sistema de patentes senil, opaco e monetariamente discriminatório.

As grandes corporações distribuidoras de conteúdos, e os detentores de cargos políticos pouco ou nada esclarecidos sobre a sociedade da informação unem-se para manter os interesses instalados das grandes distribuidoras que detêm os direitos de Copyright, atropelando os direitos dos Autores, do mesmo modo que as grandes corporações detentoras de patentes, atropelam os direitos dos  pequenos inventores. Fazem-no através da pressão sub-reptícia assente em postulados abusivos e generalistas do simples acto de  partilhar, em boa fé, a cultura e conhecimento sem quaisquer fins  lucrativos. Inclusive usando a tentativa de associação a crimes mediáticos e infames, tais como o terrorismo ou a pornografia infantil, por parte de interesses instalados sobre os detentores de cargos políticos pouco esclarecidos e/ou  com interesses associados, que promovem a introdução de leis extremamente severas e de aplicação sumária.

Contra esta tendência vigente insurge-se um conjunto de cidadãos esclarecidos e  extremamente preocupados com o rumo actual da política nacional e europeia, que se aproxima a passos largos de uma ditadura tirânica  movida por interesses económicos corporativos, e nos afasta da democracia do povo, pelo povo e para o povo, mantendo esse mesmo povo espartilhado na sua  evolução cultural e educativa como forma de o controlar.\\ 

Lutamos  pela criação do Partido Pirata Português! \\

Um novo Partido Político inspirado no emergente movimento internacional de Partidos  Piratas que aliam esforços “velejando” num mesmo sentido a favor dos  ventos de mudança na cultura mundial, contra as tempestades que atacam os nossos direitos, liberdades e garantias mais fundamentais.\\

Um Partido que lute pela Política Transparente, que invalide por inerência e na raiz qualquer tentativa de pressão corruptiva sobre os detentores de cargos políticos por parte de interesses corporativos em detrimento dos interesses dos  cidadãos.\\

Um Partido que lute pela Privacidade Pessoal, conciliando o direito de reserva da vida privada à liberdade de expressão na vida pública.\\

Um Partido que lute pela Partilha Cultural, o livre acesso às artes, à  informação, ao conhecimento em geral e a uma educação de qualidade e  de abrangência não espartilhada.\\

Um Partido que lute pela garantia ao direito de atribuição da autoria aos autores de obras culturais.\\

Um Partido que lute! Para que uma geração inteira não seja mais estigmatizada como criminosa por agir de acordo  com o que já “é parte fundamental da sua natureza”.\\

Um Partido que não seja da esquerda nem da direita nem do centro, mas que para além dos seus pilares ideológicos unificadores não irá ignorar nem desprezar o resto, contemplando para tal, e sempre, a opinião e voz de todos os seus membros em cada momento.\\

Um Partido que lute pela via legal das eleições para os parlamentos nacional e europeu onde possa bater-se por uma nova legislação que contemple todos estes objectivos e outros associados que passamos a descrever em maior detalhe.\\
\end{abstract}

%Aqui estão definidas as propriedades da listas usadas neste documento
%		Progressive - identacao (avanço) em cada nível
%		Style*, Style2*, Style3*, etc. - Estilo usado para cada nivel. O ESPAÇO NA FRENTE ANTES DA VIRGULA TAMBÉM CONTA!!!!
\newcommand{\setListProperties}{%
	\ListProperties(Hide=100, Hang=true, Progressive=3ex, 
Style*= , Style2*=-- ,Style4*=$\bullet$ , Style5*=$\circ$ , Style3*=\tiny$\blacksquare$ )
}

\section{Política Transparente}\label{Política Transparente}

\begin{easylist}[itemize]
\setListProperties
& Hoje em dia vivemos não numa República Democrática Transparente, mas numa Ditadura Oligárquica Obscura!

&& Porque as leis votadas e publicadas no Diário da Assembleia da República, são não raras vezes emendadas antes da promulgação e publicação no Diário da República, sem previamente se observar as regras do bom funcionamento da democracia.
&& Porque não se investiga como tal a corrupção, a violação e a traição à Democracia, nem nada se faz para reverter ou corrigir a situação.
&& Porque a culpa morre solteira e tal só é possível porque uma elite de oligarcas partidários se encobrem mutuamente abafando o grito de morte da Democracia.

& Defendemos que os detentores de cargos políticos sejam responsabilizados pelas suas acções!

&& A imunidade parlamentar deve ser aplicada apenas aos votos e opiniões dos seus detentores e a mais nada!
&& Defendemos que os detentores de cargos de gestão e administração pública sejam responsabilizados pelas suas acções!
&& A corrupção nunca é "pequena" e o mínimo comportamento desviante deve ser investigado e punido criminalmente e politicamente


& Para que a responsabilização seja possível é necessário que as acções dos representantes do Estado sejam transparentes!
&& A transparência permite que os cidadãos fiscalizem o estado ao invés de este controlar os cidadãos.
&& A transparência permite identificar e expor as ineficácias processuais e corrupções do sistema.
&& A transparência e o escrutínio públicos desencorajam a displicência e o desleixo.
&& A transparência e o escrutínio públicos incentivam o brio nos honestos e o receio nos corruptos.
&& A transparência desencoraja a corrupção fácil, pela facilidade de identificar os seus responsáveis.


& Defendemos uma filosofia de 'Open Government' que permita tal transparência!
&& Toda a informação relativa a decisões e processos políticos e processos financeiros na administração pública deve estar disponível e facilmente acessível a todos![b]
&&& A segurança nacional e afins devem ser apenas a excepção e não tornar-se uma regra.
&& As transmissões da Assembleia da República devem ser em sinal aberto e não restritas aos distribuidores privados!
&&& Deve haver uma expansão do âmbito das transmissões para a cobertura da política Autárquica em emissões regionais.


& Defendemos a adopção a nível nacional e europeu de um novo paradigma de Democracia real que também usamos internamente.
&& A Democracia Representativa falhou como modelo de Democracia real.
&&& Porque os “representantes” já não representam os eleitores mas sim os seus interesses e os interesses das suas estruturas partidárias.
&&& Porque os “representantes” não são escolhidos pelos eleitores, são escolhidos pelos partidos para listas que depois se votam por atacado.
&&& Porque os “representantes” eleitos sob falsas promessas não podem ser destituídos por quem os elegeu durante o período do seu mandato.
&&& Porque os “representantes” de falsas promessas, mesmo que não reeleitos num circulo, circulam para novas posições nas oligarquias partidárias, perpetuando a sua falsa “representatividade”.
&& A Democracia Directa por si só é  impraticável como modelo de Democracia real.
&&& Porque ninguém tem disponibilidade suficiente para acompanhar todas as matérias para formar uma opinião em consciência.
&&& Porque ninguém tem conhecimentos suficientes em todas as matérias de modo a poder formar uma opinião em consciência.
&&& Porque ninguém tem interesse suficiente por todas as matérias de modo a querer sequer dar uma opinião em consciência.
&&& Os Referendos rápidos e seguros recorrendo à Internet e à segurança das assinaturas digitais, mantendo o voto secreto,  já são possíveis, mas contra-indicados para as deliberações do dia a dia, onde a abstenção ganharia sempre por estas e outras razões.
&& Defendemos a adopção da Democracia Liquida participativa e real.
&&& Para que cada cidadão possa propor a qualquer momento diplomas sobre as matérias que julgue de interesse para a comunidade.
&&& Para que cada cidadão possa dar a qualquer momento o seu voto directo em qualquer diploma que deseje, independentemente de delegação prévia.
&&& Para que cada cidadão possa delegar o seu voto na globalidade ou de forma granular a um ou vários concidadãos de sua confiança.
&&& Para que cada cidadão possa retirar a sua delegação de voto a qualquer momento aos concidadãos que violem a sua confiança.
&&& Para que cada cidadão possa chamar a si a responsabilidade de participar nas decisões concretas e não ficar refém de quem decida mal por ele.

\end{easylist}

\section{Privacidade Pessoal}\label{Privacidade Pessoal}

\subsection{Liberdade de Reserva da Vida Privada}\label{Liberdade de Reserva da Vida Privada}
\begin{easylist}[itemize]
\setListProperties
& A privacidade pessoal é um direito fundamental, tal como ditado no artigo 12º da Declaração Universal dos Direitos Humanos de 1948:
&& “Ninguém sofrerá intromissões arbitrárias na sua vida privada, na sua família, no seu domicílio ou na sua correspondência, nem ataques à sua honra e reputação. Contra tais intromissões ou ataques toda a pessoa tem direito a protecção da lei.”
&&& Hoje o domicílio estende-se ao conteúdo privado dos computadores, telemóveis e demais dispositivos electrónicos pessoais!
&&& Hoje a correspondência estende-se às conversações privadas de correio electrónico, de telemóvel, de mensagens instantâneas, de redes sociais e demais formas de comunicação digital!
&& Defendemos os mesmos ideais do texto original neste novo contexto!
&&& Ignorar esta nova realidade permite violações dos Direitos Humanos baseadas em falsas noções de segurança nacional e em favor de "lobbies" corporativos de indústrias obsoletas!
&&& O combate ao terrorismo, aos crimes de alguns e aos interesses comerciais de outros não podem justificar a devassa arbitrária da vida privada de todos!
&&& Lutamos por manter o "direito a protecção da lei" contra certos desejos de violação generalizada pela lei!
&& Defendemos o artigo 26º da Constituição da República Portuguesa, número 2 "A lei estabelecerá garantias efectivas contra a obtenção e utilização abusivas, ou contrárias à dignidade humana, de informações relativas às pessoas e famílias.”
&&& Todo e qualquer tipo de excepção tem de ser previamente ordenada por um juiz com base em suspeitas solidamente fundadas!
&&& Opomo-nos ao abuso das novas tecnologias para a criação de bases de dados sobre os hábitos privados dos cidadãos sem o seu consentimento explícito!

\end{easylist}

\subsection{Liberdade de Expressão na Vida Pública}\label{Liberdade de Expressão na Vida Pública}
\begin{easylist}[itemize]
\setListProperties

& A liberdade de expressão é um direito fundamental, tal como ditado no Artigo 37.º da Constituição da República Portuguesa sobre a Liberdade de expressão e informação:
&& “1. Todos têm o direito de exprimir e divulgar livremente o seu pensamento pela palavra, pela imagem ou por qualquer outro meio, bem como o direito de informar, de se informar e de ser informados, sem impedimentos nem discriminações.”
&&& Hoje a informação é veiculada essencialmente através da Internet, pelo que se deve garantir o direito a esta “sem impedimentos nem discriminações”!
&&& Defendemos o acesso à banda larga enquanto serviço universal e acessível a todos os cidadãos, de forma a que seja possível usufruir dele sem se ser obrigado a pagar qualquer contra-partida[c] adicional!
&&& Defendemos a introdução de redes municipais sem fios nas zonas geográficas do país onde as redes das operadoras privadas não conseguem ou não querem chegar (devido a questões de rentabilidade económica)!
&& “2. O exercício destes direitos não pode ser impedido ou limitado por qualquer tipo ou forma de censura.”
&&& Defendemos estes direitos constitucionais que não poderão nunca ceder aos direitos ou interesses comerciais de quem quer que seja!
&&& Defendemos os artigos 10º e 11º da Declaração dos Direitos do Homem, contra julgamentos prévios, sumários ou por entidades não judiciais:
&&&& “10º Toda a pessoa tem direito, em plena igualdade, a que a sua causa seja equitativa e publicamente julgada por um tribunal independente e imparcial que decida dos seus direitos e obrigações ou das razões de qualquer acusação em matéria penal que contra ela seja deduzida.”
&&&& “11º Toda a pessoa acusada de um acto delituoso presume-se inocente até que a sua culpabilidade fique legalmente provada no decurso de um processo público em que todas as garantias necessárias de defesa lhe sejam asseguradas.”
&& Defendemos a liberdade de expressão dos cidadãos, associações e organizações públicas ou privadas!
&&& Opomo-nos a qualquer tentativa de suprimir ou criminalizar o uso da liberdade de expressão!


& A liberdade é um direito fundamental, tal como ditado no artigo 9º da Declaração Universal dos Direitos Humanos de 1948:
&& “Ninguém pode ser arbitrariamente preso, detido ou exilado.”
&&& Hoje o mundo reparte-se entre o real e o virtual, e a liberdade tem de ser expandida à Internet, mantendo-a neutral!
&&& Ninguém pode ser arbitrariamente privado do acesso à Internet, de ter o seu acesso diminuído ou filtrado!

\end{easylist}

\section{Partilha Cultural}\label{Partilha Cultural}

\subsection{Direitos de Autor}\label{Direitos de Autor}

\begin{easylist}[itemize]
\setListProperties
& Defendemos na sua plenitude o artigo 27º da Declaração Universal dos Direitos Humanos de 1948:
&&. “1. Toda a pessoa tem o direito de tomar parte livremente na vida cultural da comunidade, de fruir as artes e de participar no progresso científico e nos benefícios que deste resultam.”
&&& Por isso, defendemos a liberalização do uso e remistura sem fins lucrativos de obras culturais como fonte difusora de uma nova cultura!
&&& Por isso, defendemos a liberalização da partilha sem fins lucrativos[d] de obras culturais como forma promotora das mesmas e da cultura!
&& “2. Todos têm direito à protecção dos interesses morais e materiais ligados a qualquer produção científica, literária ou artística da sua autoria.”
&&& Por isso, defendemos a compensação razoável pelo uso e remistura com fins lucrativos[e] de obras culturais como uma fonte difusora de nova cultura!
&&& Por isso, defendemos a exclusividade do direito à cópia com fins lucrativos de obras culturais aos seus autores durante um prazo razoável!
&&&& Uma extensão de 70 anos após a morte dos autores não é um prazo razoável e limita o direito de fruir as artes aos seus contemporâneos!
&&&& Um prazo contado a partir da morte dos autores é um absurdo cultural que discrimina o acesso às artes consoante as posses de cada um!
&&& Por isso, defendemos um prazo contado a partir do dia de emissão da obra, e não da morte do autor, para a sua passagem ao domínio publico!
&&&& Com um período de exclusividade do copyright ao autor durante uma primeira fase desse prazo e a compensação ao mesmo durante o restante.

& Os direitos civis e constitucionais nunca podem ser postos em causa pelos direitos de autor!
&& Usar os direitos de autor como expediente para a promoção da censura de documentos incómodos de qualquer tipo deve ser proibido.

& Opomo-nos às designações e noções de "Propriedade Intelectual" e "Detentor de Copyright"! São enganadoras e nocivas para os próprios autores!
&& As entidades intelectuais não físicas, como as ideias, não são alvo de "Propriedade" mas, eventualmente, de "Autoria" com direitos morais não patrimoniais!
&& Os direitos morais e patrimoniais associados à "Autoria" não devem poder ser alienados do autor passando a ser "Propriedade" de terceiros!
&& O "Copyright" comercial, sendo um direito patrimonial da "Autoria", não deve poder ser vendido, somente licenciado durante o seu prazo!
&& Estas noções enganadoras visam o aproveitamento ilegítimo de leis sobre o direito de propriedade.

& Opomo-nos aos meios de gestão de direitos digitais intrusivos[f] (DRM) e a quaisquer privilégios legais para os mesmos, que chegam a prejudicar os consumidores!

& Opomo-nos às tentativas de restrição dos direitos do consumidor por meio de licenças de utilizador final (EULA)  encapotadas!

& Opomo-nos às taxas arbitrárias e dissimuladas sobre os meios de difusão e sobre os suportes de conteúdos audiovisuais!
&& As taxas cobradas em meios de suporte audiovisual são abusivas, exploratórias e punem indiscriminadamente todos os usos de tais meios, independentemente dos seus fins!
&& A ideia de uma taxa similar poder ser imposta nos contratos de serviços de acesso à Internet é inconcebível!
&& É hipócrita taxar o que se apregoa como ilegal! Só se pode taxar o que é legal ou então chama-se multa e não se aplica indiscriminadamente!

& Os direitos de cópia e uso não comercial de obras culturais criadas, encomendadas ou patrocinadas pelo Estado ou quaisquer entidades públicas devem ser estendidos a todos os contribuintes!
&& Os direitos de licenciamento para uso comercial a não contribuintes pertencem ao Estado para benefício de todos.
&& Os direitos de atribuição permanecem com os autores individuais.
\end{easylist}

\subsection{Patentes}\label{Patentes}

\begin{easylist}[itemize]
\setListProperties

& O sistema actual de patentes defende o exacto oposto do que esteve na sua ideia inicial!
&& A criatividade é estrangulada pelos seus custos elevados, que não protegem os pequenos inventores, apenas as grandes corporações!
&& A inovação é estrangulada, porque a sua utilização permite a supressão de tecnologias mais eficientes mas concorrentes das actuais!
&&& A tecnologia patenteada deve ser posta em prática sob pena de a patente ser anulada!
&& A inovação é estrangulada porque a sua utilização permite um modelo de negócio de extorsão legal dos especuladores de patentes!
&&& Opomo-nos à manutenção de portfolios de patentes não para a sua implementação mas apenas para processar quem a tenta.


& O sistema de patentes actual está corrompido no seu âmbito de proteger apenas invenções!
&& Opomo-nos às patentes genéticas!
&&& É um absurdo, os genes são informação, conhecimento, passíveis de descoberta, não de invenção!
&& Opomo-nos às patentes de informação biológica de qualquer espécie!
&&& Ninguém é dono da Vida nem de informação que ajude a preservar a Vida!
&&& A pesquisa deve ser patrocinada directamente pelo Estado e não indirectamente através de comparticipações em medicamentos da Segurança Social, que para além da pesquisa pagam o marketing e os lucros chorudos das corporações farmacêuticas privadas!
&& Patentear entidades não físicas ou passíveis de descoberta e não de invenção é um absurdo que mina qualquer eventual sistema de patentes!

& Opomo-nos à extensão do âmbito das patentes para além do que foi acordado na Convenção de Munique de 1973!

& Opomo-nos às patentes de algoritmos!

& Opomo-nos às patentes de software! Este deve ser protegido pelos direitos de autor!
&& O software está para os processos que implementa como a escrita para os objectos que descreve.

& Opomo-nos às patentes de modelos de apresentação de dados!

& Opomo-nos às patentes de modelos de negócio!

& Opomo-nos a qualquer tipo de patentes do trivial! Os critérios sobre o que é patenteável devem ser mais razoáveis!

& Defendemos uma redução do prazo das patentes e o encorajamento do licenciamento em oposição à exclusividade!
&& Defendemos um período de exclusividade da patente ao inventor durante uma primeira fase desse prazo e a compensação ao mesmo durante o  restante.
&&& Se o detentor da exclusividade não quiser/puder exercer esse direito durante o primeiro ano do período que lhe é reservado, passa automaticamente ao regime de período compensatório

& Defendemos um novo modelo de patentes aberto ao desenvolvimento e inovação, que compatibilize a promoção e protecção dos esforços de inovação permitindo o uso do conhecimento existente, que enalteça a evolução tecnológica a bem da sociedade, evitando que monopólios privados surjam à custa de tal modelo!

& Os direitos de uso de patentes financiadas por fundos públicos devem ser estendidos a todos os contribuintes!
&& Os direitos de licenciamento para uso comercial a não contribuintes pertencem ao Estado para benefício de todos.
&& Os direitos de atribuição permanecem com os investigadores individuais.
\end{easylist}

\subsection{Marcas Comerciais}\label{Marcas Comerciais}

\begin{easylist}[itemize]
\setListProperties
& Defendemos a manutenção e protecção[g] das marcas comerciais!
&& São um símbolo de confiança entre produtor e consumidor.
&& São um símbolo de qualidade e responsabilidade do produtor.
&& São uma garantia de fidelização de consumidores satisfeitos.
\end{easylist}

\subsection{Conteúdo Aberto - 'Open Content'}\label{Conteúdo Aberto - 'Open Content'}

\begin{easylist}[itemize]
\setListProperties
& Defendemos uma filosofia 'Open Access'!
&& O sistema de publicações científicas actuais está transformado numa máquina de fazer dinheiro às custas do Estado.
&&& Q[h]uando um investigador ou um estudante desenvolve o seu trabalho, necessita de acesso a publicações científicas, o que, na maioria dos casos, obriga o Estado a pagamentos avultados sob a forma de subscrições – com frequentes restrições de tempo e local;
&&& Quando o investigador pretende publicar o seu trabalho (pago pelo Estado) de forma acessível a todos, dentro e fora da academia, o Estado paga por essa possibilidade;
&&& Quando um estudante faz cópias do artigo paga a terceiros o copyright por algo que foi desenvolvido ao serviço da Universidade.
&& Defendemos um sistema aberto à publicação, distribuição e uso livre de artigos científicos de académicos revistos e classificados pelos seus pares.


& Defendemos uma filosofia de 'Open Data'!
&& Todos os dados passíveis de descoberta ou compilação e não de invenção devem ser disponibilizados de forma gratuita a todos!
&&& Entre este tipo de dados incluem-se mapas, compostos químicos, fórmulas matemáticas, dados médicos, genomas, biociência, biodiversidade, etc.
&&& A informação deve ser aberta, o processo que leva à sua descoberta ou compilação é que deve poder receber qualquer tipo de protecção.


& Defendemos uma filosofia de 'Open Source'!
&& Defendemos a adopção de programas informáticos de licença e código aberto nas entidades públicas!
&&& A abertura do código permite a sua revisão e validação de segurança, bem como investigação no âmbito em que se enquadra.
&&& A abertura da licença permite o melhoramento e evolução do conteúdo, bem como avultadas poupanças em licenciamentos.


& Defendemos a acessibilidade e gratuitidade do ensino básico e superior, como incentivador da cultura e direito fundamental, tal como ditado no Artigo 26.º da Declaração Universal dos Direitos Humanos de 1948, e no número 1 do Artigo 74º da Constituição da República Portuguesa
&& Defendemos a disponibilização pública de toda a documentação necessária a todos os cursos e disciplinas leccionadas a todos os níveis do ensino público.
\end{easylist}

